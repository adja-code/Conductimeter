%% Generated by Sphinx.
\def\sphinxdocclass{report}
\documentclass[letterpaper,10pt,french]{sphinxmanual}
\ifdefined\pdfpxdimen
   \let\sphinxpxdimen\pdfpxdimen\else\newdimen\sphinxpxdimen
\fi \sphinxpxdimen=.75bp\relax
\ifdefined\pdfimageresolution
    \pdfimageresolution= \numexpr \dimexpr1in\relax/\sphinxpxdimen\relax
\fi
%% let collapsible pdf bookmarks panel have high depth per default
\PassOptionsToPackage{bookmarksdepth=5}{hyperref}


\PassOptionsToPackage{warn}{textcomp}
\usepackage[utf8]{inputenc}
\ifdefined\DeclareUnicodeCharacter
% support both utf8 and utf8x syntaxes
  \ifdefined\DeclareUnicodeCharacterAsOptional
    \def\sphinxDUC#1{\DeclareUnicodeCharacter{"#1}}
  \else
    \let\sphinxDUC\DeclareUnicodeCharacter
  \fi
  \sphinxDUC{00A0}{\nobreakspace}
  \sphinxDUC{2500}{\sphinxunichar{2500}}
  \sphinxDUC{2502}{\sphinxunichar{2502}}
  \sphinxDUC{2514}{\sphinxunichar{2514}}
  \sphinxDUC{251C}{\sphinxunichar{251C}}
  \sphinxDUC{2572}{\textbackslash}
\fi
\usepackage{cmap}
\usepackage[T1]{fontenc}
\usepackage{amsmath,amssymb,amstext}
\usepackage{babel}



\usepackage{tgtermes}
\usepackage{tgheros}
\renewcommand{\ttdefault}{txtt}



\usepackage[Sonny]{fncychap}
\ChNameVar{\Large\normalfont\sffamily}
\ChTitleVar{\Large\normalfont\sffamily}
\usepackage{sphinx}

\fvset{fontsize=auto}
\usepackage{geometry}


% Include hyperref last.
\usepackage{hyperref}
% Fix anchor placement for figures with captions.
\usepackage{hypcap}% it must be loaded after hyperref.
% Set up styles of URL: it should be placed after hyperref.
\urlstyle{same}

\addto\captionsfrench{\renewcommand{\contentsname}{Sommaire:}}

\usepackage{sphinxmessages}
\setcounter{tocdepth}{1}



\title{Conductimètre}
\date{juin 29, 2025}
\release{1}
\author{Adja Condé, Nina Rubin, François métivier}
\newcommand{\sphinxlogo}{\vbox{}}
\renewcommand{\releasename}{Version}
\makeindex
\begin{document}

\ifdefined\shorthandoff
  \ifnum\catcode`\=\string=\active\shorthandoff{=}\fi
  \ifnum\catcode`\"=\active\shorthandoff{"}\fi
\fi

\pagestyle{empty}
\sphinxmaketitle
\pagestyle{plain}
\sphinxtableofcontents
\pagestyle{normal}
\phantomsection\label{\detokenize{index::doc}}


\sphinxstepscope


\chapter{Installation}
\label{\detokenize{Installation:installation}}\label{\detokenize{Installation::doc}}
\sphinxAtStartPar
Pour commencer à utiliser le conductimètre,
branchez\sphinxhyphen{}le à votre ordinateur à l’aide d’un câble
arduino.

\begin{figure}[htbp]
\centering

\noindent\sphinxincludegraphics{{images/schema_montage}.png}
\end{figure}

\sphinxAtStartPar
Dans le logiciel Spyder sur l’ordinateur, lancez le
fichier ‘ADNI\_ProgrammePython.py’. Ensuite, appuyez
sur le logo ▶︎ (indiqué par un \sphinxstylestrong{1} dans le schéma ci\sphinxhyphen{} dessous)  en haut à gauche, le programme devrait ensuite se lancer dans
la console (indiqué par un \sphinxstylestrong{2} dans le schéma ci\sphinxhyphen{} dessous).

\begin{figure}[htbp]
\centering

\noindent\sphinxincludegraphics{{image_spyder}.png}
\end{figure}

\sphinxAtStartPar
Il vous suffit ensuite de suivre les instructions dictées par le programme en rentrant vos choix
directement à la suite du programme puis en appuyant sur Entrée.
Le mode d’emploi pour l’étalonnage et les mesures se trouve
dans la suite de ce manuel.
Le nombre de valeurs prises en compte lors des calculs de
conductivité moyenne (par étalon ou par mesure) est défini par défaut, mais vous pouvez les modifier au début en
entrant ‘3’ dans l’interface d’accueil. Attention, les valeurs modifiées ne s’enregistrent pas si vous
relancez le programme.

\sphinxstepscope


\chapter{Etalonnage}
\label{\detokenize{Etalonnage:etalonnage}}\label{\detokenize{Etalonnage::doc}}
\sphinxAtStartPar
Un nouveau calibrage est conseillé à chaque nouvelle utilisation du conductimètre.
Cependant, le programme enregistre à chaque fois le dernier étalonnage en date et le réutilise si
vous n’en refaites pas.


\section{Protocole}
\label{\detokenize{Etalonnage:protocole}}\begin{enumerate}
\sphinxsetlistlabels{\arabic}{enumi}{enumii}{}{.}%
\item {} 
\sphinxAtStartPar
Entrez le nombre d’étalon que vous voulez mesurer, il est conseillé d’en faire au moins 3.

\item {} 
\sphinxAtStartPar
Indiquez la conductivité de votre premier étalon.

\end{enumerate}

\sphinxAtStartPar
\sphinxstylestrong{Remarque} : Si cette mesure correspond à 12,8 mS/cm, 1413 μS/cm ou 5000 μS/cm, les
conductivités seront corrigées selon la température de la solution. Il vous faudra alors
plonger le thermomètre dans la solution.
\begin{enumerate}
\sphinxsetlistlabels{\arabic}{enumi}{enumii}{}{.}%
\setcounter{enumi}{2}
\item {} 
\sphinxAtStartPar
Versez votre étalon dans un bécher de 50mL.

\end{enumerate}

\sphinxAtStartPar
4. Plongez le conductimètre et le thermomètre dans la solution en les immergeant bien
comme sur le schéma suivant :

\begin{figure}[htbp]
\centering

\noindent\sphinxincludegraphics{{images/schema_electrodes}.png}
\end{figure}

\sphinxAtStartPar
5. Appuyez sur Entrée pour lancer la mesure.
Cette étape peut prendre du temps.

\sphinxAtStartPar
6. L’écart type ainsi qu’un graphique vous sont affichés, il vous faut indiquer à l’ordinateur si la
mesure est assez stable.
Si la mesure n’est pas stable, elle sera effectuée une nouvelle fois.

\sphinxAtStartPar
7. Une fois que vous avez indiqué la série de mesure comme stable, vous pouvez passer à la
solution étalon suivante en répétant les étapes 2 à 6.
\begin{enumerate}
\sphinxsetlistlabels{\arabic}{enumi}{enumii}{}{.}%
\setcounter{enumi}{7}
\item {} 
\sphinxAtStartPar
Une fois tous vos étalons faits, la courbe d’étalonnage devrait s’afficher sur votre ordinateur.

\end{enumerate}

\sphinxAtStartPar
Ce calibrage est ainsi enregistré dans votre ordinateur sous le nom \sphinxstylestrong{dernier\_étalonnage.csv’}
en écrasant le dernier fichier du même nom.
\sphinxstyleemphasis{Remarque : Si vous voulez le conserver même si vous
refaite un étalonnage, il vous suffit de le renommer.}
Une fois toutes ces étapes effectuées, vous avez fini le calibrage. Vous pouvez passer aux mesures.


\section{Principe général du code python}
\label{\detokenize{Etalonnage:principe-general-du-code-python}}
\sphinxAtStartPar
Le conductimètre que nous avons construit mesure une tension en bits convertie simplement en volt. Or il existe une relation linéaire entre cette tension et la conductivité d’une solution. L’étalonnage nous permet de réaliser une régression linéaire  à partir d’échatillons dont la conductivité est connue, nous permettant par la suite de déterminer la conductivité de n’importe quel échantillon.

\sphinxAtStartPar
La conductivité varie de façon linéaire en fonction de la température. En revanche cette relation dépend de la conductivité initiale de notre échantillon. Le code python que nous avons développé intègre des corrections de température uniquement pour des solutions étalons de 12 880 us/cm, 1413us/cm et 5000 us/cm. Cela est problématique pour deux raisons. D’une part les solutions étalons mis à disposition des utilisateurs n’ont pas nécessairement ces valeurs de conductivité; D’autre part effectuer une correction de température uniquement pendant l’étalonnage suppose que la température  de la solution reste constante pendant toute la durée de la manipulation.

\begin{figure}[htbp]
\centering

\noindent\sphinxincludegraphics{{correction_temp_12880}.png}
\end{figure}

\begin{figure}[htbp]
\centering

\noindent\sphinxincludegraphics{{correction_temp_1413}.png}
\end{figure}

\begin{figure}[htbp]
\centering

\noindent\sphinxincludegraphics{{correction_temp_5000}.png}
\end{figure}

\sphinxstepscope


\chapter{Mesures}
\label{\detokenize{Mesures:mesures}}\label{\detokenize{Mesures::doc}}

\section{Mode d’emploi pour réaliser  les mesures de conductivité}
\label{\detokenize{Mesures:mode-d-emploi-pour-realiser-les-mesures-de-conductivite}}
\sphinxAtStartPar
Pour mesurer la conductivité de vos différents échantillon vous devez :
vous devez :
\begin{enumerate}
\sphinxsetlistlabels{\arabic}{enumi}{enumii}{}{.}%
\item {} 
\sphinxAtStartPar
Saisir le nombre d’échantillons à mesurer.

\end{enumerate}

\sphinxAtStartPar
2. Introduire dans un bécher de 50 mL une quantité suffisante de solution à mesurer afin
d’assurer l’immersion totale des deux électrodes du conductimètre, comme sur le schéma
suivant :

\begin{figure}[htbp]
\centering

\noindent\sphinxincludegraphics{{images/schema_electrodes}.png}
\end{figure}
\begin{enumerate}
\sphinxsetlistlabels{\arabic}{enumi}{enumii}{}{.}%
\setcounter{enumi}{2}
\item {} 
\sphinxAtStartPar
Plonger le conductimètre dans la solution.

\item {} 
\sphinxAtStartPar
Appuyer sur Entrée pour lancer la mesure.

\end{enumerate}

\sphinxAtStartPar
5. Attendre que la conductivité moyenne s’affiche sur votre écran. Cette étape peut être assez
longue et dépend du nombre de valeurs prise pour calculer la conductivité (modifiable
depuis l’accueil).
\begin{enumerate}
\sphinxsetlistlabels{\arabic}{enumi}{enumii}{}{.}%
\setcounter{enumi}{5}
\item {} 
\sphinxAtStartPar
Nettoyer l’électrode à l’eau distillée et la sécher.

\item {} 
\sphinxAtStartPar
Passer à l’échantillon suivant.

\end{enumerate}
\begin{quote}

\sphinxAtStartPar
Vos mesures ont été effectuées avec succès. Vous pouvez réaliser une nouvelle série de
\end{quote}

\sphinxAtStartPar
mesures ou procéder au rangement du matériel. N’oubliez pas de rincer les électrodes et le
thermomètre à l’eau distillée et de tout sécher avant de ranger le matériel.


\section{Principe du code Python}
\label{\detokenize{Mesures:principe-du-code-python}}
\sphinxAtStartPar
La majeure partie du travail a été effectuée pendant l’étalonnage. Cette partie consiste à mesurer un certain nombre de valeurs de conductivité en tension, à les convertir dans les bonnes unités grâce à la relation linéaire trouvée pendant l’étalonnage et à renvoyer la moyenne et l’écart type.

\sphinxstepscope


\chapter{Documentation du code python}
\label{\detokenize{Documentation:documentation-du-code-python}}\label{\detokenize{Documentation::doc}}

\section{Description de la fonction port\_connexion()}
\label{\detokenize{Documentation:description-de-la-fonction-port-connexion}}\index{port\_connexion() (dans le module lib\_conductimetre)@\spxentry{port\_connexion()}\spxextra{dans le module lib\_conductimetre}}

\begin{fulllineitems}
\phantomsection\label{\detokenize{Documentation:lib_conductimetre.port_connexion}}
\pysigstartsignatures
\pysiglinewithargsret{\sphinxcode{\sphinxupquote{lib\_conductimetre.}}\sphinxbfcode{\sphinxupquote{port\_connexion}}}{\emph{\DUrole{n}{br}\DUrole{o}{=}\DUrole{default_value}{115200}}, \emph{\DUrole{n}{portIN}\DUrole{o}{=}\DUrole{default_value}{\textquotesingle{}\textquotesingle{}}}}{}
\pysigstopsignatures
\sphinxAtStartPar
Établit la connexion au port série.
\begin{quote}\begin{description}
\sphinxlineitem{Paramètres}\begin{itemize}
\item {} 
\sphinxAtStartPar
\sphinxstyleliteralstrong{\sphinxupquote{br}} (\sphinxstyleliteralemphasis{\sphinxupquote{int}}) \textendash{} Flux de données en baud.

\item {} 
\sphinxAtStartPar
\sphinxstyleliteralstrong{\sphinxupquote{portIN}} (\sphinxstyleliteralemphasis{\sphinxupquote{string}}) \textendash{} Identifiant du port série sur lequel le script doit lire des données.

\end{itemize}

\sphinxlineitem{Renvoie}
\sphinxAtStartPar
\begin{itemize}
\item {} 
\sphinxAtStartPar
\sphinxstylestrong{port} (\sphinxstyleemphasis{string}) \textendash{} Identifiant du port série sur lequel le script doit lire des données.

\item {} 
\sphinxAtStartPar
\sphinxstylestrong{s} (\sphinxstyleemphasis{serial.tools.list\_ports\_common.ListPortInfo / string}) \textendash{} Objet Serial sur lequel on peut appliquer des fonctions d’ouverture, de lecture et de fermeture du port série affilié. En cas d’échec de connexion, “s” sera une chaîne de caractères « erreur ».

\end{itemize}


\end{description}\end{quote}

\end{fulllineitems}



\section{Description de la fonction type\_conductimetre}
\label{\detokenize{Documentation:description-de-la-fonction-type-conductimetre}}\index{type\_conductimetre() (dans le module lib\_conductimetre)@\spxentry{type\_conductimetre()}\spxextra{dans le module lib\_conductimetre}}

\begin{fulllineitems}
\phantomsection\label{\detokenize{Documentation:lib_conductimetre.type_conductimetre}}
\pysigstartsignatures
\pysiglinewithargsret{\sphinxcode{\sphinxupquote{lib\_conductimetre.}}\sphinxbfcode{\sphinxupquote{type\_conductimetre}}}{}{}
\pysigstopsignatures
\sphinxAtStartPar
Fonction qui détermine si le conductimètre possède une sonde de type K1 ou K10 en fonction du numéro de série de la carte arduino.
\begin{quote}\begin{description}
\sphinxlineitem{Renvoie}
\sphinxAtStartPar
\begin{itemize}
\item {} 
\sphinxAtStartPar
\sphinxstylestrong{type\_sonde} (\sphinxstyleemphasis{TYPE})

\item {} 
\sphinxAtStartPar
\sphinxstyleemphasis{Entier qui vaut 1 Si la sonde du conductimètre est de type K1 et 10 si la sonde est de type K10.}

\end{itemize}


\end{description}\end{quote}

\end{fulllineitems}



\section{Description de la fonction mesure\_etalonnage()}
\label{\detokenize{Documentation:description-de-la-fonction-mesure-etalonnage}}\index{mesure\_etalonnage() (dans le module lib\_conductimetre)@\spxentry{mesure\_etalonnage()}\spxextra{dans le module lib\_conductimetre}}

\begin{fulllineitems}
\phantomsection\label{\detokenize{Documentation:lib_conductimetre.mesure_etalonnage}}
\pysigstartsignatures
\pysiglinewithargsret{\sphinxcode{\sphinxupquote{lib\_conductimetre.}}\sphinxbfcode{\sphinxupquote{mesure\_etalonnage}}}{\emph{\DUrole{n}{nbr\_mesure\_par\_etalon}}, \emph{\DUrole{n}{conductimeter}}, \emph{\DUrole{n}{C25}}, \emph{\DUrole{n}{type\_conductimeter}}}{}
\pysigstopsignatures
\sphinxAtStartPar
Fonction permettant de faire une série de mesures à haute fréquence, et de la représenter sur un graphique
\begin{quote}\begin{description}
\sphinxlineitem{Paramètres}\begin{itemize}
\item {} 
\sphinxAtStartPar
\sphinxstyleliteralstrong{\sphinxupquote{nbr\_mesure\_par\_etalon}} (\sphinxstyleliteralemphasis{\sphinxupquote{int}}) \textendash{} nombre de mesures par étalon, décidé dans les valeurs par défaut.

\item {} 
\sphinxAtStartPar
\sphinxstyleliteralstrong{\sphinxupquote{Returns}} \textendash{} 

\item {} 
\sphinxAtStartPar
\sphinxstyleliteralstrong{\sphinxupquote{conductivite}} (\sphinxstyleliteralemphasis{\sphinxupquote{int}}) \textendash{} Conductivité moyenne à température ambiante

\item {} 
\sphinxAtStartPar
\sphinxstyleliteralstrong{\sphinxupquote{Ecart\_type\_conductivite}} (\sphinxstyleliteralemphasis{\sphinxupquote{int}}) \textendash{} Ecart\sphinxhyphen{}type associé à la valeur de conductivite moyenne

\item {} 
\sphinxAtStartPar
\sphinxstyleliteralstrong{\sphinxupquote{Tension\_etalon}} (\sphinxstyleliteralemphasis{\sphinxupquote{int}}) \textendash{} 

\item {} 
\sphinxAtStartPar
\sphinxstyleliteralstrong{\sphinxupquote{ambiante}} (\sphinxstyleliteralemphasis{\sphinxupquote{Tension moyenne mesurée à température}}) \textendash{} 

\item {} 
\sphinxAtStartPar
\sphinxstyleliteralstrong{\sphinxupquote{Ecart\_type\_tension}} (\sphinxstyleliteralemphasis{\sphinxupquote{int}}) \textendash{} Ecart\_type associé à la valeure de tension mesurée en dortie de la sonde conductimétrique.

\item {} 
\sphinxAtStartPar
\sphinxstyleliteralstrong{\sphinxupquote{Temperature\_etalon}} (\sphinxstyleliteralemphasis{\sphinxupquote{int}}) \textendash{} Température moyenne de la solution

\item {} 
\sphinxAtStartPar
\sphinxstyleliteralstrong{\sphinxupquote{Ecart\_type\_temperature}} (\sphinxstyleliteralemphasis{\sphinxupquote{int}}) \textendash{} Écart type associé à la valeur moyenne de conductivité.

\item {} 
\sphinxAtStartPar
\sphinxstyleliteralstrong{\sphinxupquote{alpha}} (\sphinxstyleliteralemphasis{\sphinxupquote{int}}) \textendash{} Coefficient de correction de tepérature à 25 °C calculé à partir des données présentes sur les solutions tampons Hanna Instruments

\item {} 
\sphinxAtStartPar
\sphinxstyleliteralstrong{\sphinxupquote{\sphinxhyphen{}\sphinxhyphen{}\sphinxhyphen{}\sphinxhyphen{}\sphinxhyphen{}\sphinxhyphen{}\sphinxhyphen{}}} \textendash{} 

\end{itemize}

\end{description}\end{quote}

\end{fulllineitems}



\section{Description de la fonction Etalonnage\_K1()}
\label{\detokenize{Documentation:description-de-la-fonction-etalonnage-k1}}\index{Etalonnage\_K1() (dans le module lib\_conductimetre)@\spxentry{Etalonnage\_K1()}\spxextra{dans le module lib\_conductimetre}}

\begin{fulllineitems}
\phantomsection\label{\detokenize{Documentation:lib_conductimetre.Etalonnage_K1}}
\pysigstartsignatures
\pysiglinewithargsret{\sphinxcode{\sphinxupquote{lib\_conductimetre.}}\sphinxbfcode{\sphinxupquote{Etalonnage\_K1}}}{\emph{\DUrole{n}{nbr\_etalon}}, \emph{\DUrole{n}{nbr\_mesure\_par\_etalon}}, \emph{\DUrole{n}{conductimeter}}, \emph{\DUrole{n}{type\_conductimeter}}}{}
\pysigstopsignatures
\sphinxAtStartPar
Fonction qui permet d’étalonner les sondes de type K10 grâce à une régression polyfit si le nombre d’étalons vaut 2  ou gâce à la méthode des moindres carrés et qui enregistre les données automatiquement dans un fichier en format csv et les courbes d’étalonnage au format choisit par l’utilisateur.
\begin{quote}\begin{description}
\sphinxlineitem{Paramètres}\begin{itemize}
\item {} 
\sphinxAtStartPar
\sphinxstyleliteralstrong{\sphinxupquote{nbr\_etalon}} (\sphinxstyleliteralemphasis{\sphinxupquote{int}}) \textendash{} Nombre de solutions étalons

\item {} 
\sphinxAtStartPar
\sphinxstyleliteralstrong{\sphinxupquote{nbr\_mesure\_par\_etalon}} (\sphinxstyleliteralemphasis{\sphinxupquote{int}}) \textendash{} Nombre de mesures par étalon (200 par défaut à modifier si besoin)

\item {} 
\sphinxAtStartPar
\sphinxstyleliteralstrong{\sphinxupquote{conductimeter}} (\sphinxstyleliteralemphasis{\sphinxupquote{serial.tools.list\_ports\_common.ListPortInfo / string}}) \textendash{} Objet Serial sur lequel on peut appliquer des fonctions d’ouverture, de lecture et de fermeture du port série affilié. En cas d’échec de connexion, “s” sera une chaîne de caractères « erreur ».

\item {} 
\sphinxAtStartPar
\sphinxstyleliteralstrong{\sphinxupquote{type\_conductimeter}} (\sphinxstyleliteralemphasis{\sphinxupquote{int}}) \textendash{} Entier qui vaut 1 si la sonde est de type K1 et 10 si la sonde est de type K10

\end{itemize}

\sphinxlineitem{Renvoie}
\sphinxAtStartPar
\begin{itemize}
\item {} 
\sphinxAtStartPar
\sphinxstylestrong{a} (\sphinxstyleemphasis{int}) \textendash{} Coefficient directeur de la régression linéaire C=f(U)

\item {} 
\sphinxAtStartPar
\sphinxstylestrong{b} (\sphinxstyleemphasis{int}) \textendash{} Coefficient directeur de la régression linéaire C=f(U)

\end{itemize}


\end{description}\end{quote}

\end{fulllineitems}



\section{Description de la fonction Etalonnage\_K10()}
\label{\detokenize{Documentation:description-de-la-fonction-etalonnage-k10}}\index{Etalonnage\_K10() (dans le module lib\_conductimetre)@\spxentry{Etalonnage\_K10()}\spxextra{dans le module lib\_conductimetre}}

\begin{fulllineitems}
\phantomsection\label{\detokenize{Documentation:lib_conductimetre.Etalonnage_K10}}
\pysigstartsignatures
\pysiglinewithargsret{\sphinxcode{\sphinxupquote{lib\_conductimetre.}}\sphinxbfcode{\sphinxupquote{Etalonnage\_K10}}}{\emph{\DUrole{n}{nbr\_etalon}}, \emph{\DUrole{n}{nbr\_mesure\_par\_etalon}}, \emph{\DUrole{n}{conductimeter}}, \emph{\DUrole{n}{type\_conductimeter}}}{}
\pysigstopsignatures
\sphinxAtStartPar
Fonction qui permet d’étalonner les sondes de type K10 grâce à une régression polyfit et qui enregistre les données automatiquement dans un fichier en format csv et les coubres d’étalonnage au format choisit par l’utilisateur.
\begin{quote}\begin{description}
\sphinxlineitem{Paramètres}\begin{itemize}
\item {} 
\sphinxAtStartPar
\sphinxstyleliteralstrong{\sphinxupquote{nbr\_etalon}} (\sphinxstyleliteralemphasis{\sphinxupquote{int}}) \textendash{} Nombre de solutions étalons

\item {} 
\sphinxAtStartPar
\sphinxstyleliteralstrong{\sphinxupquote{nbr\_mesure\_par\_etalon}} (\sphinxstyleliteralemphasis{\sphinxupquote{int}}) \textendash{} 
\sphinxAtStartPar
Nombre de mesures par solution étalons ( 200 par défaut à modifier si besoin dans les paramètres par défaut)
conductimeter : serial.tools.list\_ports\_common.ListPortInfo / string
\begin{quote}

\sphinxAtStartPar
Objet Serial sur lequel on peut appliquer des fonctions d’ouverture, de lecture et de fermeture du port série affilié. En cas d’échec de connexion, “s” sera une chaîne de caractères « erreur ».
\end{quote}


\item {} 
\sphinxAtStartPar
\sphinxstyleliteralstrong{\sphinxupquote{type\_conductimeter}} (\sphinxstyleliteralemphasis{\sphinxupquote{int}}) \textendash{} Entier qui vaut 1 si le conductimètre possède une sonde K1 et 10 si la sonde est de type K10

\end{itemize}

\sphinxlineitem{Renvoie}
\sphinxAtStartPar
\begin{itemize}
\item {} 
\sphinxAtStartPar
\sphinxstylestrong{a} (\sphinxstyleemphasis{int}) \textendash{} Coefficient directeur de la courbe d’étalonnage”

\item {} 
\sphinxAtStartPar
\sphinxstylestrong{b} (\sphinxstyleemphasis{int}) \textendash{} Ordonnée à l’origine de la courbe d’étalonnage

\end{itemize}


\end{description}\end{quote}

\end{fulllineitems}



\section{Description de la fonction Mesures\_K1()}
\label{\detokenize{Documentation:description-de-la-fonction-mesures-k1}}\index{Mesures\_K1() (dans le module lib\_conductimetre)@\spxentry{Mesures\_K1()}\spxextra{dans le module lib\_conductimetre}}

\begin{fulllineitems}
\phantomsection\label{\detokenize{Documentation:lib_conductimetre.Mesures_K1}}
\pysigstartsignatures
\pysiglinewithargsret{\sphinxcode{\sphinxupquote{lib\_conductimetre.}}\sphinxbfcode{\sphinxupquote{Mesures\_K1}}}{\emph{\DUrole{n}{a}}, \emph{\DUrole{n}{b}}, \emph{\DUrole{n}{nbr\_mesure\_par\_echantillon}}, \emph{\DUrole{n}{conductimeter}}}{}
\pysigstopsignatures
\sphinxAtStartPar
Fonction qui convertit la tension mesurée par le conductimètre en conductivité, applique une correction de température afin d’obtenir la conductivité à 25°C, et enregistrerles données dans un fichier pour une sonde K1”
\begin{quote}\begin{description}
\sphinxlineitem{Paramètres}\begin{itemize}
\item {} 
\sphinxAtStartPar
\sphinxstyleliteralstrong{\sphinxupquote{a}} (\sphinxstyleliteralemphasis{\sphinxupquote{int}}) \textendash{} Coefficient directeur de la courbe d’étalonage

\item {} 
\sphinxAtStartPar
\sphinxstyleliteralstrong{\sphinxupquote{b}} (\sphinxstyleliteralemphasis{\sphinxupquote{int}}) \textendash{} Ordonnée à l’origine de la courbe d’étalonnage

\item {} 
\sphinxAtStartPar
\sphinxstyleliteralstrong{\sphinxupquote{nbr\_mesure\_par\_echantillon}} (\sphinxstyleliteralemphasis{\sphinxupquote{int}}) \textendash{} Nombre de mesures par échantillon

\item {} 
\sphinxAtStartPar
\sphinxstyleliteralstrong{\sphinxupquote{conductimeter}} (\sphinxstyleliteralemphasis{\sphinxupquote{serial.tools.list\_ports\_common.ListPortInfo / string}}) \textendash{} Objet Serial sur lequel on peut appliquer des fonctions d’ouverture, de lecture et de fermeture du port série affilié. En cas d’échec de connexion, “s” sera une chaîne de caractères « erreur ».

\end{itemize}

\sphinxlineitem{Renvoie}
\sphinxAtStartPar
\begin{itemize}
\item {} 
\sphinxAtStartPar
\sphinxstylestrong{conductivite} (\sphinxstyleemphasis{int}) \textendash{} Conductivité moyenne de la solution  à Température ambiante

\item {} 
\sphinxAtStartPar
\sphinxstylestrong{C25} (\sphinxstyleemphasis{int}) \textendash{} Conductivité moyenne de la solution à 25 °C

\item {} 
\sphinxAtStartPar
\sphinxstylestrong{temperature} (\sphinxstyleemphasis{int}) \textendash{} Température moyenne de la solution

\item {} 
\sphinxAtStartPar
\sphinxstylestrong{date} (\sphinxstyleemphasis{datetime}) \textendash{} Date et heure de la mesure

\end{itemize}


\end{description}\end{quote}

\end{fulllineitems}



\section{Description de la fonction Mesures\_K10()}
\label{\detokenize{Documentation:description-de-la-fonction-mesures-k10}}\index{Mesures\_K10() (dans le module lib\_conductimetre)@\spxentry{Mesures\_K10()}\spxextra{dans le module lib\_conductimetre}}

\begin{fulllineitems}
\phantomsection\label{\detokenize{Documentation:lib_conductimetre.Mesures_K10}}
\pysigstartsignatures
\pysiglinewithargsret{\sphinxcode{\sphinxupquote{lib\_conductimetre.}}\sphinxbfcode{\sphinxupquote{Mesures\_K10}}}{\emph{\DUrole{n}{a}}, \emph{\DUrole{n}{b}}, \emph{\DUrole{n}{nbr\_mesure\_par\_echantillon}}, \emph{\DUrole{n}{conductimeter}}}{}
\pysigstopsignatures
\sphinxAtStartPar
Fonction qui convertit la tension mesurée par le conductimètre en conductivité, applique une correction de température afin d’obtenir la conductivité à 25°C, et enregistrerles données dans un fichier pour une sonde K1”
\begin{quote}\begin{description}
\sphinxlineitem{Paramètres}\begin{itemize}
\item {} 
\sphinxAtStartPar
\sphinxstyleliteralstrong{\sphinxupquote{a}} (\sphinxstyleliteralemphasis{\sphinxupquote{int}}) \textendash{} Coefficient directeur de la courbe d’étalonage

\item {} 
\sphinxAtStartPar
\sphinxstyleliteralstrong{\sphinxupquote{b}} (\sphinxstyleliteralemphasis{\sphinxupquote{int}}) \textendash{} Ordonnée à l’origine de la courbe d’étalonnage

\item {} 
\sphinxAtStartPar
\sphinxstyleliteralstrong{\sphinxupquote{nbr\_mesure\_par\_echantillon}} (\sphinxstyleliteralemphasis{\sphinxupquote{int}}) \textendash{} Nombre de mesures par échantillon

\item {} 
\sphinxAtStartPar
\sphinxstyleliteralstrong{\sphinxupquote{conductimeter}} (\sphinxstyleliteralemphasis{\sphinxupquote{serial.tools.list\_ports\_common.ListPortInfo / string}}) \textendash{} Objet Serial sur lequel on peut appliquer des fonctions d’ouverture, de lecture et de fermeture du port série affilié. En cas d’échec de connexion, “s” sera une chaîne de caractères « erreur ».

\end{itemize}

\sphinxlineitem{Renvoie}
\sphinxAtStartPar
\begin{itemize}
\item {} 
\sphinxAtStartPar
\sphinxstylestrong{conductivite} (\sphinxstyleemphasis{int}) \textendash{} Conductivité moyenne de la solution  à Température ambiante

\item {} 
\sphinxAtStartPar
\sphinxstylestrong{C25} (\sphinxstyleemphasis{int}) \textendash{} Conductivité moyenne de la solution à 25 °C

\item {} 
\sphinxAtStartPar
\sphinxstylestrong{temperature} (\sphinxstyleemphasis{int}) \textendash{} Température moyenne de la solution

\item {} 
\sphinxAtStartPar
\sphinxstylestrong{date} (\sphinxstyleemphasis{datetime}) \textendash{} Date et heure de la mesure

\end{itemize}


\end{description}\end{quote}

\end{fulllineitems}



\section{Description de la fonction correction\_temperature\_etalonnage()}
\label{\detokenize{Documentation:description-de-la-fonction-correction-temperature-etalonnage}}\index{Mesures\_K1() (dans le module lib\_conductimetre)@\spxentry{Mesures\_K1()}\spxextra{dans le module lib\_conductimetre}}

\begin{fulllineitems}
\phantomsection\label{\detokenize{Documentation:id0}}
\pysigstartsignatures
\pysiglinewithargsret{\sphinxcode{\sphinxupquote{lib\_conductimetre.}}\sphinxbfcode{\sphinxupquote{Mesures\_K1}}}{\emph{\DUrole{n}{a}}, \emph{\DUrole{n}{b}}, \emph{\DUrole{n}{nbr\_mesure\_par\_echantillon}}, \emph{\DUrole{n}{conductimeter}}}{}
\pysigstopsignatures
\sphinxAtStartPar
Fonction qui convertit la tension mesurée par le conductimètre en conductivité, applique une correction de température afin d’obtenir la conductivité à 25°C, et enregistrerles données dans un fichier pour une sonde K1”
\begin{quote}\begin{description}
\sphinxlineitem{Paramètres}\begin{itemize}
\item {} 
\sphinxAtStartPar
\sphinxstyleliteralstrong{\sphinxupquote{a}} (\sphinxstyleliteralemphasis{\sphinxupquote{int}}) \textendash{} Coefficient directeur de la courbe d’étalonage

\item {} 
\sphinxAtStartPar
\sphinxstyleliteralstrong{\sphinxupquote{b}} (\sphinxstyleliteralemphasis{\sphinxupquote{int}}) \textendash{} Ordonnée à l’origine de la courbe d’étalonnage

\item {} 
\sphinxAtStartPar
\sphinxstyleliteralstrong{\sphinxupquote{nbr\_mesure\_par\_echantillon}} (\sphinxstyleliteralemphasis{\sphinxupquote{int}}) \textendash{} Nombre de mesures par échantillon

\item {} 
\sphinxAtStartPar
\sphinxstyleliteralstrong{\sphinxupquote{conductimeter}} (\sphinxstyleliteralemphasis{\sphinxupquote{serial.tools.list\_ports\_common.ListPortInfo / string}}) \textendash{} Objet Serial sur lequel on peut appliquer des fonctions d’ouverture, de lecture et de fermeture du port série affilié. En cas d’échec de connexion, “s” sera une chaîne de caractères « erreur ».

\end{itemize}

\sphinxlineitem{Renvoie}
\sphinxAtStartPar
\begin{itemize}
\item {} 
\sphinxAtStartPar
\sphinxstylestrong{conductivite} (\sphinxstyleemphasis{int}) \textendash{} Conductivité moyenne de la solution  à Température ambiante

\item {} 
\sphinxAtStartPar
\sphinxstylestrong{C25} (\sphinxstyleemphasis{int}) \textendash{} Conductivité moyenne de la solution à 25 °C

\item {} 
\sphinxAtStartPar
\sphinxstylestrong{temperature} (\sphinxstyleemphasis{int}) \textendash{} Température moyenne de la solution

\item {} 
\sphinxAtStartPar
\sphinxstylestrong{date} (\sphinxstyleemphasis{datetime}) \textendash{} Date et heure de la mesure

\end{itemize}


\end{description}\end{quote}

\end{fulllineitems}



\chapter{Indices and tables}
\label{\detokenize{index:indices-and-tables}}\begin{itemize}
\item {} 
\sphinxAtStartPar
\DUrole{xref,std,std-ref}{genindex}

\item {} 
\sphinxAtStartPar
\DUrole{xref,std,std-ref}{modindex}

\item {} 
\sphinxAtStartPar
\DUrole{xref,std,std-ref}{search}

\end{itemize}



\renewcommand{\indexname}{Index}
\printindex
\end{document}