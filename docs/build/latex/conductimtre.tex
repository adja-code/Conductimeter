%% Generated by Sphinx.
\def\sphinxdocclass{report}
\documentclass[letterpaper,10pt,french]{sphinxmanual}
\ifdefined\pdfpxdimen
   \let\sphinxpxdimen\pdfpxdimen\else\newdimen\sphinxpxdimen
\fi \sphinxpxdimen=.75bp\relax
\ifdefined\pdfimageresolution
    \pdfimageresolution= \numexpr \dimexpr1in\relax/\sphinxpxdimen\relax
\fi
%% let collapsible pdf bookmarks panel have high depth per default
\PassOptionsToPackage{bookmarksdepth=5}{hyperref}


\PassOptionsToPackage{warn}{textcomp}
\usepackage[utf8]{inputenc}
\ifdefined\DeclareUnicodeCharacter
% support both utf8 and utf8x syntaxes
  \ifdefined\DeclareUnicodeCharacterAsOptional
    \def\sphinxDUC#1{\DeclareUnicodeCharacter{"#1}}
  \else
    \let\sphinxDUC\DeclareUnicodeCharacter
  \fi
  \sphinxDUC{00A0}{\nobreakspace}
  \sphinxDUC{2500}{\sphinxunichar{2500}}
  \sphinxDUC{2502}{\sphinxunichar{2502}}
  \sphinxDUC{2514}{\sphinxunichar{2514}}
  \sphinxDUC{251C}{\sphinxunichar{251C}}
  \sphinxDUC{2572}{\textbackslash}
\fi
\usepackage{cmap}
\usepackage[T1]{fontenc}
\usepackage{amsmath,amssymb,amstext}
\usepackage{babel}



\usepackage{tgtermes}
\usepackage{tgheros}
\renewcommand{\ttdefault}{txtt}



\usepackage[Sonny]{fncychap}
\ChNameVar{\Large\normalfont\sffamily}
\ChTitleVar{\Large\normalfont\sffamily}
\usepackage{sphinx}

\fvset{fontsize=auto}
\usepackage{geometry}


% Include hyperref last.
\usepackage{hyperref}
% Fix anchor placement for figures with captions.
\usepackage{hypcap}% it must be loaded after hyperref.
% Set up styles of URL: it should be placed after hyperref.
\urlstyle{same}

\addto\captionsfrench{\renewcommand{\contentsname}{Sommaire:}}

\usepackage{sphinxmessages}
\setcounter{tocdepth}{1}



\title{Conductimètre}
\date{juin 26, 2025}
\release{1}
\author{Adja Condé, Nina Rubin, François métivier}
\newcommand{\sphinxlogo}{\vbox{}}
\renewcommand{\releasename}{Version}
\makeindex
\begin{document}

\ifdefined\shorthandoff
  \ifnum\catcode`\=\string=\active\shorthandoff{=}\fi
  \ifnum\catcode`\"=\active\shorthandoff{"}\fi
\fi

\pagestyle{empty}
\sphinxmaketitle
\pagestyle{plain}
\sphinxtableofcontents
\pagestyle{normal}
\phantomsection\label{\detokenize{index::doc}}


\sphinxstepscope


\chapter{Installation}
\label{\detokenize{Installation:installation}}\label{\detokenize{Installation::doc}}
\sphinxAtStartPar
Pour commencer à utiliser le conductimètre,
branchez\sphinxhyphen{}le à votre ordinateur à l’aide d’un câble
arduino.

\begin{figure}[htbp]
\centering

\noindent\sphinxincludegraphics{{schema_montage}.png}
\end{figure}

\sphinxAtStartPar
Dans le logiciel Spyder sur l’ordinateur, lancez le
fichier ‘ADNI\_ProgrammePython.py’. Ensuite, appuyez
sur le logo ▶︎ (indiqué par un \sphinxstylestrong{1} dans le schéma ci\sphinxhyphen{} dessous)  en haut à gauche, le programme devrait ensuite se lancer dans
la console (indiqué par un \sphinxstylestrong{2} dans le schéma ci\sphinxhyphen{} dessous).

\begin{figure}[htbp]
\centering

\noindent\sphinxincludegraphics{{image_spyder}.png}
\end{figure}

\sphinxAtStartPar
Il vous suffit ensuite de suivre les instructions dictées par le programme en rentrant vos choix
directement à la suite du programme puis en appuyant sur Entrée.
Le mode d’emploi pour l’étalonnage et les mesures se trouve
dans la suite de ce manuel.
Le nombre de valeurs prises en compte lors des calculs de
conductivité moyenne (par étalon ou par mesure) est défini par défaut, mais vous pouvez les modifier au début en
entrant ‘3’ dans l’interface d’accueil. Attention, les valeurs modifiées ne s’enregistrent pas si vous
relancez le programme.

\sphinxstepscope


\chapter{Etalonnage}
\label{\detokenize{Etalonnage:etalonnage}}\label{\detokenize{Etalonnage::doc}}
\sphinxAtStartPar
Un nouveau calibrage est conseillé à chaque nouvelle utilisation du conductimètre.
Cependant, le programme enregistre à chaque fois le dernier étalonnage en date et le réutilise si
vous n’en refaites pas.


\section{Protocole}
\label{\detokenize{Etalonnage:protocole}}\begin{enumerate}
\sphinxsetlistlabels{\arabic}{enumi}{enumii}{}{.}%
\item {} 
\sphinxAtStartPar
Entrez le nombre d’étalon que vous voulez mesurer, il est conseillé d’en faire au moins 3.

\item {} 
\sphinxAtStartPar
Indiquez la conductivité de votre premier étalon.

\end{enumerate}

\sphinxAtStartPar
\sphinxstylestrong{Remarque} : Si cette mesure correspond à 12,8 mS/cm, 1413 μS/cm ou 5000 μS/cm, les
conductivités seront corrigées selon la température de la solution. Il vous faudra alors
plonger le thermomètre dans la solution.
\begin{enumerate}
\sphinxsetlistlabels{\arabic}{enumi}{enumii}{}{.}%
\setcounter{enumi}{2}
\item {} 
\sphinxAtStartPar
Versez votre étalon dans un bécher de 50mL.

\end{enumerate}

\sphinxAtStartPar
4. Plongez le conductimètre et le thermomètre dans la solution en les immergeant bien
comme sur le schéma suivant :

\begin{figure}[htbp]
\centering

\noindent\sphinxincludegraphics{{schema_electrodes}.png}
\end{figure}

\sphinxAtStartPar
5. Appuyez sur Entrée pour lancer la mesure.
Cette étape peut prendre du temps.

\sphinxAtStartPar
6. L’écart type ainsi qu’un graphique vous sont affichés, il vous faut indiquer à l’ordinateur si la
mesure est assez stable.
Si la mesure n’est pas stable, elle sera effectuée une nouvelle fois.

\sphinxAtStartPar
7. Une fois que vous avez indiqué la série de mesure comme stable, vous pouvez passer à la
solution étalon suivante en répétant les étapes 2 à 6.
\begin{enumerate}
\sphinxsetlistlabels{\arabic}{enumi}{enumii}{}{.}%
\setcounter{enumi}{7}
\item {} 
\sphinxAtStartPar
Une fois tous vos étalons faits, la courbe d’étalonnage devrait s’afficher sur votre ordinateur.

\end{enumerate}

\sphinxAtStartPar
Ce calibrage est ainsi enregistré dans votre ordinateur sous le nom \sphinxstylestrong{dernier\_étalonnage.csv’}
en écrasant le dernier fichier du même nom.
\sphinxstyleemphasis{Remarque : Si vous voulez le conserver même si vous
refaite un étalonnage, il vous suffit de le renommer.}
Une fois toutes ces étapes effectuées, vous avez fini le calibrage. Vous pouvez passer aux mesures.


\section{Principe général du code python}
\label{\detokenize{Etalonnage:principe-general-du-code-python}}
\sphinxAtStartPar
Le conductimètre que nous avons construit mesure une tension en bits convertie simplement en volt. Or il existe une relation linéaire entre cette tension et la conductivité d’une solution. L’étalonnage nous permet de réaliser une régression linéaire  à partir d’échatillons dont la conductivité est connue, nous permettant par la suite de déterminer la conductivité de n’importe quel échantillon.

\sphinxAtStartPar
La conductivité varie de façon linéaire en fonction de la température. En revanche cette relation dépend de la conductivité initiale de notre échantillon. Le code python que nous avons développé intègre des corrections de température uniquement pour des solutions étalons de 12 880 us/cm, 1413us/cm et 5000 us/cm. Cela est problématique pour deux raisons. D’une part les solutions étalons mis à disposition des utilisateurs n’ont pas nécessairement ces valeurs de conductivité; D’autre part effectuer une correction de température uniquement pendant l’étalonnage suppose que la température  de la solution reste constante pendant toute la durée de la manipulation.

\begin{figure}[htbp]
\centering

\noindent\sphinxincludegraphics{{correction_temp_12880}.png}
\end{figure}

\begin{figure}[htbp]
\centering

\noindent\sphinxincludegraphics{{correction_temp_1413}.png}
\end{figure}

\begin{figure}[htbp]
\centering

\noindent\sphinxincludegraphics{{correction_temp_5000}.png}
\end{figure}

\sphinxstepscope


\chapter{Mesures}
\label{\detokenize{Mesures:mesures}}\label{\detokenize{Mesures::doc}}

\section{Mode d’emploi pour réaliser  les mesures de conductivité}
\label{\detokenize{Mesures:mode-d-emploi-pour-realiser-les-mesures-de-conductivite}}
\sphinxAtStartPar
Pour mesurer la conductivité de vos différents échantillon vous devez :
vous devez :
\begin{enumerate}
\sphinxsetlistlabels{\arabic}{enumi}{enumii}{}{.}%
\item {} 
\sphinxAtStartPar
Saisir le nombre d’échantillons à mesurer.

\end{enumerate}

\sphinxAtStartPar
2. Introduire dans un bécher de 50 mL une quantité suffisante de solution à mesurer afin
d’assurer l’immersion totale des deux électrodes du conductimètre, comme sur le schéma
suivant :

\begin{figure}[htbp]
\centering

\noindent\sphinxincludegraphics{{schema_electrodes}.png}
\end{figure}
\begin{enumerate}
\sphinxsetlistlabels{\arabic}{enumi}{enumii}{}{.}%
\setcounter{enumi}{2}
\item {} 
\sphinxAtStartPar
Plonger le conductimètre dans la solution.

\item {} 
\sphinxAtStartPar
Appuyer sur Entrée pour lancer la mesure.

\end{enumerate}

\sphinxAtStartPar
5. Attendre que la conductivité moyenne s’affiche sur votre écran. Cette étape peut être assez
longue et dépend du nombre de valeurs prise pour calculer la conductivité (modifiable
depuis l’accueil).
\begin{enumerate}
\sphinxsetlistlabels{\arabic}{enumi}{enumii}{}{.}%
\setcounter{enumi}{5}
\item {} 
\sphinxAtStartPar
Nettoyer l’électrode à l’eau distillée et la sécher.

\item {} 
\sphinxAtStartPar
Passer à l’échantillon suivant.

\end{enumerate}
\begin{quote}

\sphinxAtStartPar
Vos mesures ont été effectuées avec succès. Vous pouvez réaliser une nouvelle série de
\end{quote}

\sphinxAtStartPar
mesures ou procéder au rangement du matériel. N’oubliez pas de rincer les électrodes et le
thermomètre à l’eau distillée et de tout sécher avant de ranger le matériel.


\section{Principe du code Python}
\label{\detokenize{Mesures:principe-du-code-python}}
\sphinxAtStartPar
La majeure partie du travail a été effectuée pendant l’étalonnage. Cette partie consiste à mesurer un certain nombre de valeurs de conductivité en tension, à les convertir dans les bonnes unités grâce à la relation linéaire trouvée pendant l’étalonnage et à renvoyer la moyenne et l’écart type.

\sphinxstepscope


\chapter{Documentation du code python}
\label{\detokenize{Documentation:documentation-du-code-python}}\label{\detokenize{Documentation::doc}}

\section{Description de la fonction setup}
\label{\detokenize{Documentation:description-de-la-fonction-setup}}\index{setup() (dans le module ADNI\_ProgrammePython)@\spxentry{setup()}\spxextra{dans le module ADNI\_ProgrammePython}}

\begin{fulllineitems}
\phantomsection\label{\detokenize{Documentation:ADNI_ProgrammePython.setup}}
\pysigstartsignatures
\pysiglinewithargsret{\sphinxcode{\sphinxupquote{ADNI\_ProgrammePython.}}\sphinxbfcode{\sphinxupquote{setup}}}{}{}
\pysigstopsignatures
\sphinxAtStartPar
Paramètre d’initialisation de la carte Arduino.
\begin{quote}\begin{description}
\sphinxlineitem{Renvoie}
\sphinxAtStartPar
\sphinxstylestrong{arduino} \textendash{} Localisation de la carte arduino du conductimètre.

\sphinxlineitem{Type renvoyé}
\sphinxAtStartPar
TYPE

\end{description}\end{quote}

\end{fulllineitems}



\section{Description de la fonction mesure\_etalonnage}
\label{\detokenize{Documentation:description-de-la-fonction-mesure-etalonnage}}\index{mesure\_etalonnage() (dans le module ADNI\_ProgrammePython)@\spxentry{mesure\_etalonnage()}\spxextra{dans le module ADNI\_ProgrammePython}}

\begin{fulllineitems}
\phantomsection\label{\detokenize{Documentation:ADNI_ProgrammePython.mesure_etalonnage}}
\pysigstartsignatures
\pysiglinewithargsret{\sphinxcode{\sphinxupquote{ADNI\_ProgrammePython.}}\sphinxbfcode{\sphinxupquote{mesure\_etalonnage}}}{\emph{\DUrole{n}{nbr\_mesure\_par\_etalon}}}{}
\pysigstopsignatures
\sphinxAtStartPar
Fonction permettant de faire plein de mesure à haute fréquence puis les mettant dans un graphique pour vérifier qu’elles sont stables”
\begin{quote}\begin{description}
\sphinxlineitem{Paramètres}
\sphinxAtStartPar
\sphinxstyleliteralstrong{\sphinxupquote{nbr\_mesure\_par\_etalon}} (\sphinxstyleliteralemphasis{\sphinxupquote{int}}) \textendash{} nombre de mesures par étalon, décidé dans les valeurs par défaut.

\sphinxlineitem{Renvoie}
\sphinxAtStartPar
\sphinxstylestrong{list\_tension\_etalon} \textendash{} Liste des tensions mesurées pendant l’étalonnage.

\sphinxlineitem{Type renvoyé}
\sphinxAtStartPar
list

\end{description}\end{quote}

\end{fulllineitems}



\section{Description de la fonction d’étalonnage}
\label{\detokenize{Documentation:description-de-la-fonction-d-etalonnage}}\index{Etalonnage() (dans le module ADNI\_ProgrammePython)@\spxentry{Etalonnage()}\spxextra{dans le module ADNI\_ProgrammePython}}

\begin{fulllineitems}
\phantomsection\label{\detokenize{Documentation:ADNI_ProgrammePython.Etalonnage}}
\pysigstartsignatures
\pysiglinewithargsret{\sphinxcode{\sphinxupquote{ADNI\_ProgrammePython.}}\sphinxbfcode{\sphinxupquote{Etalonnage}}}{\emph{\DUrole{n}{nbr\_etalon}}, \emph{\DUrole{n}{nbr\_mesure\_par\_etalon}}}{}
\pysigstopsignatures
\sphinxAtStartPar
Fonction qui, pour le nombre d’étalon indiqué en argument, mesure la tension, trouve la corrélation entre Tensino et conductivité, puis trace le graphique et renvoie la droite d’étalonnage
\begin{quote}\begin{description}
\sphinxlineitem{Paramètres}\begin{itemize}
\item {} 
\sphinxAtStartPar
\sphinxstyleliteralstrong{\sphinxupquote{nbr\_etalon}} (\sphinxstyleliteralemphasis{\sphinxupquote{int}}) \textendash{} Nombre d’étalon pour l’étalonnage.

\item {} 
\sphinxAtStartPar
\sphinxstyleliteralstrong{\sphinxupquote{nbr\_mesure\_par\_etalon}} (\sphinxstyleliteralemphasis{\sphinxupquote{int}}) \textendash{} Nombre de mesure à faire pour chaque étalon.

\end{itemize}

\sphinxlineitem{Renvoie}
\sphinxAtStartPar
\sphinxstylestrong{droite} \textendash{} Droite d’étalonnage sous la forme ax + b.

\sphinxlineitem{Type renvoyé}
\sphinxAtStartPar
numpy.poly1d

\end{description}\end{quote}

\end{fulllineitems}



\section{Description de la fonction de mesures}
\label{\detokenize{Documentation:description-de-la-fonction-de-mesures}}\index{Mesures() (dans le module ADNI\_ProgrammePython)@\spxentry{Mesures()}\spxextra{dans le module ADNI\_ProgrammePython}}

\begin{fulllineitems}
\phantomsection\label{\detokenize{Documentation:ADNI_ProgrammePython.Mesures}}
\pysigstartsignatures
\pysiglinewithargsret{\sphinxcode{\sphinxupquote{ADNI\_ProgrammePython.}}\sphinxbfcode{\sphinxupquote{Mesures}}}{\emph{\DUrole{n}{nbr\_echantillon}}, \emph{\DUrole{n}{droite}}, \emph{\DUrole{n}{nbr\_mesure\_par\_echantillon}}}{}
\pysigstopsignatures
\sphinxAtStartPar
Fonction pour faire les mesures de conductivité et de température et les stocker dans un fichier.
\begin{quote}\begin{description}
\sphinxlineitem{Paramètres}\begin{itemize}
\item {} 
\sphinxAtStartPar
\sphinxstyleliteralstrong{\sphinxupquote{nbr\_echantillon}} (\sphinxstyleliteralemphasis{\sphinxupquote{int}}) \textendash{} Nombre d’échantillons à mesurer dans la série.

\item {} 
\sphinxAtStartPar
\sphinxstyleliteralstrong{\sphinxupquote{droite}} (\sphinxstyleliteralemphasis{\sphinxupquote{numpy.poly1d}}) \textendash{} Equation de droite du calibrage permettant de lier la tension mesurer par le conductimètre et la conductivité réelle.

\item {} 
\sphinxAtStartPar
\sphinxstyleliteralstrong{\sphinxupquote{nbr\_mesure\_par\_echantillon}} (\sphinxstyleliteralemphasis{\sphinxupquote{int}}) \textendash{} Nombre de mesure sur lesquelles les conductivité et température moyennes seront calculées.

\end{itemize}

\sphinxlineitem{Type renvoyé}
\sphinxAtStartPar
None.

\end{description}\end{quote}

\end{fulllineitems}



\section{Description de la fonction graph\_conductimeter}
\label{\detokenize{Documentation:description-de-la-fonction-graph-conductimeter}}\index{graph\_conductimeter() (dans le module ADNI\_ProgrammePython)@\spxentry{graph\_conductimeter()}\spxextra{dans le module ADNI\_ProgrammePython}}

\begin{fulllineitems}
\phantomsection\label{\detokenize{Documentation:ADNI_ProgrammePython.graph_conductimeter}}
\pysigstartsignatures
\pysiglinewithargsret{\sphinxcode{\sphinxupquote{ADNI\_ProgrammePython.}}\sphinxbfcode{\sphinxupquote{graph\_conductimeter}}}{}{}
\pysigstopsignatures
\sphinxAtStartPar
Fonction qui permet de tracer à partir du fichier data\_conductivité.csv la température et la conductivité en fonction du temps.
\begin{quote}\begin{description}
\sphinxlineitem{Type renvoyé}
\sphinxAtStartPar
None.

\end{description}\end{quote}

\end{fulllineitems}



\chapter{Indices and tables}
\label{\detokenize{index:indices-and-tables}}\begin{itemize}
\item {} 
\sphinxAtStartPar
\DUrole{xref,std,std-ref}{genindex}

\item {} 
\sphinxAtStartPar
\DUrole{xref,std,std-ref}{modindex}

\item {} 
\sphinxAtStartPar
\DUrole{xref,std,std-ref}{search}

\end{itemize}



\renewcommand{\indexname}{Index}
\printindex
\end{document}